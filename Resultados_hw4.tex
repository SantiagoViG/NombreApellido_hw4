\documentclass{article}
\usepackage{graphicx}
\graphicspath{ {images/} }
 
\begin{document}
A continuacion se encuentran las graficas para el caso 1, que consiste en una pllas condiciones iniciales son tales que toda la placa esta a T= 50C excepto por un rectangulo de 20cm×10cm que esta a T= 100C
\begin{center} 
\includegraphics[scale=0.5]{Caso1 Condiciones abiertas 0s.png} 
\vspace{0.5cm}
\includegraphics[scale=0.5]{Caso1 Condiciones abiertas 100s.png} 
\vspace{0.5cm}
\includegraphics[scale=0.5]{Caso1 Condiciones abiertas 2500s.png} 
\vspace{0.5cm}

\includegraphics[scale=0.5]{Caso1 Condiciones fijas 0s.png} 
\vspace{0.5cm}
\includegraphics[scale=0.5]{Caso1 Condiciones fijas 100s.png} 
\vspace{0.5cm}
\includegraphics[scale=0.5]{Caso1 Condiciones fijas 2500s.png} 
\vspace{0.5cm}

\includegraphics[scale=0.5]{Caso1 Condiciones periodicas 0s.png} 
\vspace{0.5cm}
\includegraphics[scale=0.5]{Caso1 Condiciones periodicas 100s.png} 
\vspace{0.5cm}
\includegraphics[scale=0.5]{Caso1 Condiciones periodicas 2500s.png} 
\vspace{0.5cm}
\end{center}

Ahora se presentan las graficas del caso dos, donde se rempla la placa por un generador de calor.

\begin{center} 
\includegraphics[scale=0.5]{Caso2 Condiciones abiertas 0s.png} 
\vspace{0.5cm}
\includegraphics[scale=0.5]{Caso2 Condiciones abiertas 100s.png} 
\vspace{0.5cm}
\includegraphics[scale=0.5]{Caso2 Condiciones abiertas 2500s.png} 
\vspace{0.5cm}

\includegraphics[scale=0.5]{Caso2 Condiciones fijas 0s.png} 
\vspace{0.5cm}
\includegraphics[scale=0.5]{Caso2 Condiciones fijas 100s.png} 
\vspace{0.5cm}
\includegraphics[scale=0.5]{Caso2 Condiciones fijas 2500s.png} 
\vspace{0.5cm}

\includegraphics[scale=0.5]{Caso2 Condiciones periodicas 0s.png} 
\vspace{0.5cm}
\includegraphics[scale=0.5]{Caso2 Condiciones periodicas 100s.png} 
\vspace{0.5cm}
\includegraphics[scale=0.5]{Caso2 Condiciones periodicas 2500s.png} 
\vspace{0.5cm}
\end{center}
\end{document}
